\documentclass[a4paper,11pt]{article}

\usepackage[utf8]{inputenc}

\usepackage{graphicx}
\usepackage{caption}
\usepackage{subcaption}

\usepackage{pgfplots}
\usepackage{float}
\usepackage{hyperref}
\usepackage{soul}
\hypersetup{
    colorlinks=true, % Enable colored links
    linkcolor=black, % Color for internal links
    urlcolor=black,  % Color for external links
    citecolor=black, % Color for citation links
    pdfborder={0 0 0}, % Remove border around links
}
\newcommand{\underlinehref}[2]{%
    \href{#1}{\ul{#2}}%
}
\pgfplotsset{compat=1.18}


\usepackage{minted}

\begin{document}

    \title{
        \textbf{Trees in C}
    }
    \author{Péter Herczku}
    \date{Fall 2024}

    \maketitle

    \section*{Introduction}

    The task is to implement the tree data structure as well as its operations.
    We also needed to investigate the depth first traversal method when printing out our tree in order.

    I completed the assignment using the C programming language.

    \section*{Trees}

    A tree is similar to a linked list but a node can link to more than one nodes.
    In this assignment we looked at binary trees which means that each node has at most two branches.
    The reason we call this a tree is because we have a top node which is the root and all the other branches come from that same element.
    
    Sorted binary trees require their elements to be comparable. In the assignment I used integers because they satisfy this condition and are easy to work with.

    \subsection*{Methods}

    \begin{itemize}
        \item \textbf{add(tree* tr, int item)} - adds an element to the tree while keeping it sorted
        \item \textbf{lookup(tree* tr, int item)} - looks up a node which the value {\tt item}
     \end{itemize}

     We also need to have methods that create a tree and destroy them. Let's take a look at our implementation now.
     For the add method we can choose to implement it recursively or not. First, we investigate the recursive approach:

     \begin{minted}{c}
void add_node(node* nd, int value) {
    if (nd->value == value) return;
    if (value < nd->value) {
        if (nd->left == NULL) {
            nd->left = construct_node(value);
        } else {
            add_node(nd->left, value);
        }
    } else {
        if (nd->right == NULL) {
            nd->right = construct_node(value);
        } else {
            add_node(nd->right, value);
        }
    }
}
     \end{minted}

     This algorithm the finds the empty spot that is right for our new node, while keeping it sorted.
     Since the branches of a tree are still trees we can do it recursively, but C cannot recognize that our branches are subtrees, therefore we need to create another method that takes a tree as an argument:

     \begin{minted}{c}
void add_recursive(tree *tr, int value) {
    if(tr->root == NULL) tr->root = construct_node(value);
    else add_node(tr->root, value);
}
     \end{minted}

\end{document}
