\documentclass[a4paper,11pt]{article}

\usepackage[utf8]{inputenc}

\usepackage{graphicx}
\usepackage{caption}
\usepackage{subcaption}

\usepackage{pgfplots}
\usepackage{float}
\usepackage{hyperref}
\usepackage{soul}
\hypersetup{
    colorlinks=true, % Enable colored links
    linkcolor=black, % Color for internal links
    urlcolor=black,  % Color for external links
    citecolor=black, % Color for citation links
    pdfborder={0 0 0}, % Remove border around links
}
\newcommand{\underlinehref}[2]{%
    \href{#1}{\ul{#2}}%
}
\pgfplotsset{compat=1.18}


\usepackage{minted}

\begin{document}

    \title{
        \textbf{T9 in C}
    }
    \author{Péter Herczku}
    \date{Fall 2024}

    \maketitle

    \section*{Introduction}

    The task is to implement the T9 decoder/encoder.
    I completed the assignment using the C programming language

    \section*{History}

    In the old days when you wanted to send a text message to your friend, you had to use the numpad of your mobile phone.
    If you pressed the button '1' once, you got an A. 
    If you pressed it twice you got a B and so on and so fourth.
    This method was slow, and the T9 system made it easier, as it could guess the words you meant without having to press a button multiple times.
    You only needed to press the buttons once, and the phone corrected your message.
    Of course there were mispredictions, but in most cases this algorithm worked well.

    In this report we will be discussing how to implement it in C, for the most common Swedish words.

    \section*{Character encoding}

    In C we can use the {\tt char} type to declare an ASCII character.
    However, in Swedish, there are three characters that are not in the ASCII character codec: Ö, Å and Ä.
    Therefore, they are represented in UTF-8, which means we need to include two libraries to handle them: {\tt wchar.h} and {\tt locale.h}.
    A regular {\tt char} is represented using 1 byte, but these {\tt wchar\_t} type variables are represented using 2 bytes.

    \section*{Dictionary}

    We need to the dictionary from {\tt kelly.txt} into our C program.
    We create our own character encoding which goes from 0 to 26.
    Our data structure will be a tree, where each node has 27 branches.

    \begin{minted}{c}
typedef struct node {
    bool valid;
    struct node *next[27];
} node;

typedef struct trie {
  node *root;
} trie;
    \end{minted}

    When we are adding a word, we split it by its characters and put it in our tree the following way:
    
    Let's say we have the word: apple, which is encoded as: 0 15 15 11 4. 
    This means that the root node has a pointer to a node, where the 15th position points to another node.
    The 15th position of this new node points to a node where the 11th position points to a node and so on until the 4th position points to a node where {\tt valid = true}.
    This sounds very complicated, but we could look at it as the word would be represented from top to bottom in the tree.
    Looking at the code with comments is probably easier to understand.

    We can implement this in C recursively:

    \begin{minted}{c}
node *add(node *nd, wchar_t *rest) {
    if (nd == NULL) {
        // if the node doesn't exist, we create a new
        // node, where we set  the branches to be NULL
        nd = (node*)malloc(sizeof(node));
        for (int i = 0; i < 27; i++ ) {
            nd->next[i] = NULL;
        }
        nd->valid = false;
    }
    if (rest[0] == '\n') {
        // if res[0] == '\n', it means we reached the end
        // of the string, therefore we can set valid to be true
        nd->valid = true;
    } else {
        // if not, we calculate the code in our own codec,
        // and go one level deeper while setting up the pointer
        // to hold the node we return from add
        int c = code((int)*rest);
        nd->next[c] = add(nd->next[c], (rest + 1));
    } 
    // in the end, we return the node
    return nd;
}
    \end{minted}

    \section*{Utility functions}

    We need to set up some utility functions in order to make our lives easier.
    We can have a function:
    \begin{itemize}
        \item that takes a {\tt wchar\_t } character and encodes it into our own codec
        \item that takes code and decodes it into a {\tt wchar\_t} character
        \item that takes key (1, 2 ... 9) and turns it into an index (0, 1 ... 8)
        \item that a {\tt wchar\_t} character and turns it into a key (1, 2 ... 9)
     \end{itemize}

     \section*{Decoding}

     Now that we are all set up, we can finally implement the decode algorithm.
     Since multiple codes can be encoded the same way, we need to return all possible words that could have been encoded.
     We can do it recursively:

     \begin{minted}{c}
void collect(node* nd, char* fullKey, wchar_t* sequence,
 int length, wchar_t** possible_words, int* count) {
    // the current key (1, 2 ... 9) we are at right now
    char key = fullKey[length];
    // if the key is '\0', we have reached the encoded string's end
    if (key == '\0') {
        // if it's valid, we add a terminating 0 to the end of our decoded string 
        // and copy it over into a final structure
        if (nd->valid) {
            sequence[length] = L'\0';
            wchar_t* final = (wchar_t*)malloc(sizeof(wchar_t)*(length + 1));
            for(int i = 0; i <= length; i++) {
                final[i] = sequence[i];
            }
            // we also store this final structure in the array where we store 
            // all the possible combinations
            possible_words[*count] = final;
            // and increment count, so if we found another combination it will
            // hold the proper value
            (*count)++;
        }
        return;
    }   

    // if its not terminated, we can be sure that the character is a valid key,
    // therefore we can turn it into an index
    int idx = key_to_index(key);
    // we examine all possible codes that correcspond to the index
    for (int i = 0; i < 3; i++) {
        int code = idx*3 + i;
        node* branch = nd->next[code];
        // if the branch is NULL it means, it's a dead end,
        // so we exit the iteration
        if(branch == NULL) continue;
        // we decode the code to a wchar_t character
        wchar_t nextChar = code_to_character(code);
        // we create a new structure, copy over the current sequence of letters
        // and append the new character as well
        wchar_t* final = (wchar_t*)malloc(sizeof(wchar_t)*(length + 1));
        for(int i = 0; i <= length; i++) {
            final[i] = sequence[i];
        }
        final[length] = nextChar;
        // we call the collect method again, but we increment the length by one,
        // so it will examine the next key
        collect(branch, fullKey, final, length + 1, possible_words, count);
    }
} 
     \end{minted}

     We also need to create another function, that takes in the original sequence of keys we want to decode:

     \begin{minted}{c}
wchar_t** decode(trie* dict, char* key) {
    wchar_t** possible_words = (wchar_t**)malloc(sizeof(wchar_t*)*1000);
    int count = 0;
    // we pass the root and a sufficiently big array to hold the results
    collect(dict->root, key, (wchar_t*)malloc(sizeof(wchar_t)*1), 0, possible_words, &count);
    return possible_words;
}
     \end{minted}

     \section*{Testing}

     We can test the program by checking if it can decode any valid sequence of keys.
     I created another function that takes a {\tt wchar\_t} string and encodes it.
     Then we can use the result of this function to decode and see if it matches our original word.
     After throughoutly testing the program, we can say that it's working properly.
    
     \section*{GitHub}
    I have uploaded the full project to \underlinehref{https://github.com/peterherczku/ID1021/tree/main/assignment-8-B}{my github repository}, where we can find the code used to make this report.


\end{document}