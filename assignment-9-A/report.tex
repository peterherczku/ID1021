\documentclass[a4paper,11pt]{article}

\usepackage[utf8]{inputenc}

\usepackage{graphicx}
\usepackage{caption}
\usepackage{subcaption}

\usepackage{pgfplots}
\usepackage{float}
\usepackage{hyperref}
\usepackage{soul}
\hypersetup{
    colorlinks=true, % Enable colored links
    linkcolor=black, % Color for internal links
    urlcolor=black,  % Color for external links
    citecolor=black, % Color for citation links
    pdfborder={0 0 0}, % Remove border around links
}
\newcommand{\underlinehref}[2]{%
    \href{#1}{\ul{#2}}%
}
\pgfplotsset{compat=1.18}


\usepackage{minted}

\begin{document}

    \title{
        \textbf{Dijsktra's algorithm in C}
    }
    \author{Péter Herczku}
    \date{Fall 2024}

    \maketitle

    \section*{Introduction}

    The task is to implement Dijkstra's algorithm, and calculate various distances between cities in Europe.

    I completed the assignment using the C programming language.

    \section*{Data structures}

    We will be using the same data structures that we used in the last assignment, however we need to extend them a little.
    I decided to implement all data structures in a seperate file {\tt utils.c}, so our {\tt main.c} is more readable.
    We need to add a data structure called {\tt path}, which holds the current stop, the previous stop and the total distance from the origin.

    \begin{minted}{c}
struct path {
    city* city;
    city* previous_stop;
    int distance;
} typedef path;
    \end{minted}

    I have also imported the priority queue and dynamic array data structures which will be useful during the implementation of the algorithm.
    A priority queue is similar to a queue but when we dequeue, we dequeue the path with the least total distance from the origin.

    \section*{Dijkstra's algorithm}

    We start exploring the root's neighbours and their distances.
    We enqueue them, and then start exploring the closest unvisited node: we update the shortest known distance to this node (if it's actually shorter than the current one).
    By repeating this process, we should end up with the shortest path to our destination (or we visit all nodes meaning we couldn't find the destination).

    Now let's see the code:

    \begin{minted}{c}
int dijkstra(city* from, city* to, int* doneSize) {
    priority_queue *queue = create_queue();
    // we enqueue the "root" path
    enqueue(queue, create_path(NULL, from, 0));

    dynamic_array* arr = create_array();

    while (!is_empty(queue)) {
        // we dequeue the path with the shortest distance so far
        path* p = dequeue(queue);
        city* c = p->city;
        int hashValue = hash(c->name, MOD);

        // we return the distance when we reach the destination
        if(c == to) return p->distance;
        // if we have already visited the node we ignore it
        if(!contains(arr, hashValue)) {
            // we add the city to the already visited list
            add(arr, hashValue);
            // we increment the amount of the visited nodes
            *(doneSize)+=1;

            for(int i = 0; i < c->connections_size; i++) {
                // we iterate through the neighbours of the current city
                connection* conn = c->connections[i];
                int connHashValue = hash(conn->destination->name, MOD);
                // if we have already visited it, we ignore it
                if(contains(arr, connHashValue)) continue;
                // if we haven't, we create a new path and enqueue it,
                // while also incrementing the distance
                int res = p->distance + conn->minutes;
                path* newp = create_path(c, conn->destination, res);
                enqueue(queue, newp);
            }
        }
    }

    // we return -1, if there is no path between the two cities
    return -1;
}
    \end{minted}

    \section*{Benchmarking}

    Let's do the same benchmarks that we did in the previous assignment, so we can compare them:

    \begin{table}[H]
        \begin{center}
            \begin{tabular}{c|c|c|c}
                \textbf{From} & \textbf{To} & \textbf{Visited cities} & \textbf{Runtime (us)}\\
                \hline
                Malmö  &   Göteborg     &  19 &     11.10\\
                Göteborg &    Stockholm    &  37 &     32.90\\
                Malmö &    Stockholm    &  34 &     36.80\\
                Stockholm &    Sundsvall    &  49 &     37.50\\
                Stockholm &    Umeå    &  56 &     42.30\\
                Göteborg &    Sundsvall    &  56 &     42.90\\
                Sundsvall &    Umeå    &  3 &     2.40\\
                Göteborg &    Umeå    &  65 &     64.50\\
            \end{tabular}
            \caption{Benchmarking Dijkstra's algorithm}
            \label{tab:table1}
        \end{center}
    \end{table}

    As we can see our runtime is much better (the results are in microseconds) compared to our last attempt.
    We can even calculate the distance between European cities using this algorithm.
    Let's benchmark the distance between Budapest and 12 European cities:

    \begin{table}[H]
        \begin{center}
            \begin{tabular}{c|c|c}
                \textbf{To} & \textbf{Visited cities} & \textbf{Runtime (us)}\\
                \hline
                Stockholm     &  106 &     128.5\\
                Barcelona    &  72 &     169.8\\
                München   &  4 &     405\\
                Wien    &  1 &     1.00\\
                Bukarest    &  38 &     41.6\\
                Paris    &  32 &     40.40\\
                Berlin    &  14 &     8.7\\
                London    &  36 &     23.80\\
                Oslo    &  107 &     95.2\\
                Amsterdam    &  30 &     19.8\\
                Bologna    &  35 &     22.90\\
                Naples    &  52 &     36.40\\
            \end{tabular}
            \caption{Benchmarking Dijkstra's algorithm again}
            \label{tab:table2}
        \end{center}
    \end{table}

    In general, the fewer cities we visit the better our runtime is.
    However, this is not always true, because the runtime highly depends on our data set.
    We have much more connections in Sweden than in other countries, increasing runtime.
    We could say that the time complexity depends on the structure of the graph and its density, rather than the number of cities we have to go through.
    It's hard to derive a time complexity for this algorithm due to this factor, but we can be sure that it relates to $n$.

    \section*{GitHub}
    I have uploaded the full project to \underlinehref{https://github.com/peterherczku/ID1021/tree/main/assignment-9-A}{my github repository}, where we can find the code used to make this report.


\end{document}